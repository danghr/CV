%%%%%%%%%%%%%%%%%%%%%%%%%%%%%%%%%%%%%%%%%
% Plasmati Graduate CV
% LaTeX Template
% Version 1.0 (24/3/13)
%
% This template has been downloaded from:
% http://www.LaTeXTemplates.com
%
% Original author:
% Alessandro Plasmati (alessandro.plasmati@gmail.com)
%
% License:
% CC BY-NC-SA 3.0 (http://creativecommons.org/licenses/by-nc-sa/3.0/)
%
% Important note:
% This template needs to be compiled with XeLaTeX.
% The main document font is called Fontin and can be downloaded for free
% from here: http://www.exljbris.com/fontin.html
%
%%%%%%%%%%%%%%%%%%%%%%%%%%%%%%%%%%%%%%%%%

%----------------------------------------------------------------------------------------
%	PACKAGES AND OTHER DOCUMENT CONFIGURATIONS
%----------------------------------------------------------------------------------------

\documentclass[a4paper,10pt]{ctexart} % Default font size and paper size

\usepackage{fontspec} % For loading fonts
\defaultfontfeatures{Mapping=tex-text}
\setmainfont{Helvetica Neue}
\setsansfont{Times New Roman}
\setmonofont{Courier}
\setCJKmainfont{Noto Sans CJK SC}
\setCJKsansfont{Noto Serif CJK SC}
\setCJKmonofont{Noto Sans Mono CJK SC}

\usepackage{xunicode,xltxtra,url,parskip} % Formatting packages

\usepackage[usenames,dvipsnames]{xcolor} % Required for specifying custom colors

% \usepackage[big]{layaureo} % Margin formatting of the A4 page, an alternative to layaureo can be \usepackage{fullpage}
% To reduce the height of the top margin uncomment: \addtolength{\voffset}{-1.3cm}
% \usepackage{fullpage}
\usepackage[hmargin=0.5in, vmargin=0.5in]{geometry}

\usepackage{hyperref} % Required for adding links	and customizing them
\definecolor{linkcolour}{rgb}{0,0,0} % Link color
\hypersetup{colorlinks,breaklinks,urlcolor=linkcolour,linkcolor=linkcolour} % Set link colors throughout the document

\usepackage{titlesec} % Used to customize the \section command
\titleformat{\section}{\Large\raggedright}{}{0em}{}[\titlerule] % Text formatting of sections
\titlespacing{\section}{0pt}{3pt}{6pt} % Spacing around sections
\titleformat{\subsection}{\large\raggedright}{ -}{0pt}{} % Text formatting of subsections
\titlespacing{\subsection}{0pt}{0pt}{3pt} % Spacing around subsections

\usepackage{makecell}
\usepackage{array}
\usepackage{tabularx}
\usepackage{longtable}

\usepackage{fontawesome5}

\usepackage{enumitem}
\setenumerate[1]{itemsep=3pt,partopsep=0pt,parsep=0pt,topsep=3pt}
\setitemize[1]{itemsep=3pt,partopsep=0pt,parsep=0pt,topsep=3pt}
\setdescription{itemsep=3pt,partopsep=0pt,parsep=0pt,topsep=3pt}

\begin{document}

\pagestyle{empty} % Removes page numbering

\font\fb=''[cmr10]'' % Change the font of the \LaTeX command under the skills section

% Sections that should not be cut in two pages
\newenvironment{keepsection}{\par\noindent\minipage{\textwidth}}{\endminipage\par}

% Tilde in date range
\newcommand{\datetlide}{\textasciitilde \ }


%----------------------------------------------------------------------------------------
%	NAME AND CONTACT INFORMATION
%----------------------------------------------------------------------------------------

\begin{keepsection}
\begin{center}
    {\LARGE 党浩然} \\
    \faIcon{envelope} \  \href{mailto:danghr@outlook.com}{danghr@outlook.com} \  / \  \faIcon{desktop} \  \href{https://www.danghr.com}{danghr.com} \\
\end{center}
\end{keepsection}


%----------------------------------------------------------------------------------------
%	EDUCATION
%----------------------------------------------------------------------------------------
\begin{keepsection}

\section{教育背景}

\begin{itemize}
    \item \textbf{中国科学院计算技术研究所} \hfill 2022-09至今 \\
    电子信息专业 \, 工程硕士在读 \quad {\small 导师:范东睿(研究员) \quad GPA:3.81/4} \hfill 中国 \, 北京 \\
    {\small 学籍属于中国科学院大学,学历学位均由该校授予。}
    \item \textbf{上海科技大学} \hfill 2018-09 \datetlide 2022-06 \\
    计算机科学与技术专业 \, 工学学士 \quad {\small GPA:3.50/4 \quad 专业排名:55/194(28.4\%)} \hfill 中国 \, 上海
\end{itemize}

\end{keepsection}


%----------------------------------------------------------------------------------------
%	RESEARCH & PROJECTS
%----------------------------------------------------------------------------------------

\begin{keepsection}
\section{科研和项目经历}

\subsection{发表论文}
    \begin{itemize}
        \item \underline{H Dang}, M Wu, M Yan, X Ye, and D Fan, ``\textbf{GDL-GNN: Applying GPU Dataloading of Large Datasets for Graph Neural Network Inference},'' accepted by \textit{Euro-Par 2024}.
        \item \underline{H Dang}, C Ye, Y Hu, and C Wang, ``\textbf{NobLSM: An LSM-tree with Non-blocking Writes for SSDs},'' \textit{DAC 2022}.
        \item D Wang, M Yan, Y Teng, D Han, \underline{H Dang}, X Ye, and D Fan, ``\textbf{A Transfer Learning Framework for High-accurate Cross-workload Design Space Exploration of CPU},'' \textit{ICCAD 2023}.
    \end{itemize}
\end{keepsection}

\begin{keepsection}
\subsection{研究经历}
\begin{itemize}
    \item \textbf{图计算组} \quad 中国科学院计算技术研究所 \, 高通量计算机研究中心 \hfill 2021-12至今 \\
    {\small 课题组长:严明玉(副研究员)} \hfill 中国 \, 北京
    \begin{small}
        \begin{itemize}
            \item 针对多GPU平台上的GNN推理加速问题进行研究,提出并实现了一套基于分区的优化GPU数据加载方法,性能比基线提高了59.9\%。(参见GDL-GNN)
            \item 完成了一项图神经网络在多GPU架构下训练加速的研究,通过多线程数据加载方法将训练时间减半。此工作获得了上科大信息学院优秀本科毕业设计奖项。
        \end{itemize}
    \end{small}

    \item \textbf{吐司实验室} \quad 上海科技大学 \, 信息科学与技术学院 \hfill 2021-07 \datetlide 2022-02 \\
    {\small 实验室PI:王春东(助理教授、研究员)} \hfill 中国 \, 上海
    \begin{small}
        \begin{itemize}
            \item 负责NobLSM项目中数据库部分的设计与实现,借助一版经过修改的ext4文件系统推迟旧数据的删除,以在保证数据一致性的前提下减少文件系统~\texttt{sync}~。完成该项目的性能测试工作,结果表明其性能优于SOTA LSM树数据库。(参见NobLSM)
        \end{itemize}
    \end{small}

    \item \textbf{态度研究实验室} \quad 上海科技大学 \, 创业与管理学院 \hfill 2020-09 \datetlide 2021-09 \\
    {\small 实验室PI:杨丽凤(副教授、研究员)} \hfill 中国 \, 上海
    \begin{small}
        \begin{itemize}
            \item 参与书籍《信息科学技术伦理与道德》的出版准备工作。提供出版社比选建议,并参与了样章编写。
        \end{itemize}
    \end{small}
\end{itemize}
\end{keepsection}

% \begin{keepsection}
% \subsection{Projects}
% \begin{itemize}
%     \item \textbf{``One Student One Chip'' Project}
% \end{itemize}
% \end{keepsection}

\begin{keepsection}
\subsection{教学经历}

    \begin{itemize}
        \item \textbf{操作系统I}(助教) \hfill 上海科技大学 \quad 2021-2022学年秋学期
    \end{itemize}
\end{keepsection}


%----------------------------------------------------------------------------------------
%	SKILLS
%----------------------------------------------------------------------------------------

\begin{keepsection}

\section{技能}
\begin{itemize}
    \item \textbf{编程语言和框架:} 熟练使用C/C++、Python,熟悉PyTorch、\LaTeX,了解Verilog、Shell脚本
    \item \textbf{语言:} 中文普通话(母语)、英语(流利)
        \begin{itemize}
            \item \textbf{托福网考} \quad 得分:107 {\small (阅读:29 \quad 听力:27 \quad 口语:23 \quad 写作:28)} \hfill 2024-04
            \item \textbf{GRE普通考试} \quad 得分:328 {\small (文字推理:158 \quad 数量推理:170 \quad 分析性写作:4)} \hfill 2022-08
            \item \textbf{大学英语六级考试} \quad 得分:572 {\small (听力:198 \quad 阅读:211 \quad 写作:163)} \hfill 2021-06
        \end{itemize}
\end{itemize}

\end{keepsection}


%----------------------------------------------------------------------------------------
%	CERTIFICATES
%----------------------------------------------------------------------------------------

\begin{keepsection}

\section{证书和荣誉称号}
\begin{itemize}
    \item \textbf{上海科技大学信息学院2022年优秀毕业设计} \hfill \quad 2022-06
    \item \textbf{上海科技大学2018级本科生产业实践优秀个人称号} \hfill \quad 2021-06
    \item \textbf{上海科技大学2019年本科生社会实践优秀个人称号} \hfill \quad 2019-11
\end{itemize}

\end{keepsection}


%----------------------------------------------------------------------------------------
%	Service
%----------------------------------------------------------------------------------------

\begin{keepsection}

\section{志愿服务}
\begin{itemize}
    \item \textbf{中科院计算所第20届公众科学日} \quad {\small 担任外场摄像工作。} \hfill 中科院计算所 \quad 2024-05
    \item \textbf{“向阳花活动”—海淀医院志愿者} \quad {\small 协助服务中心解答患者问题。} \hfill 中科院计算所 \quad 2023-12
    \item \textbf{2020年新生入学} \quad {\small 研究生新生行李提取引导。} \hfill 上海科技大学 \quad 2020-09
    \item \textbf{2020年校园开放日} \quad {\small 参加设备检查及试卷下载工作。} \hfill 上海科技大学 \quad 2020-07
    \item \textbf{2019年本科招生} \quad {\small 参加了北京招生组的咨询宣传工作。} \hfill 上海科技大学 \quad 2019-06
\end{itemize}

\end{keepsection}

\end{document}

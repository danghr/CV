%%%%%%%%%%%%%%%%%%%%%%%%%%%%%%%%%%%%%%%%%
% Plasmati Graduate CV
% LaTeX Template
% Version 1.0 (24/3/13)
%
% This template has been downloaded from:
% http://www.LaTeXTemplates.com
%
% Original author:
% Alessandro Plasmati (alessandro.plasmati@gmail.com)
%
% License:
% CC BY-NC-SA 3.0 (http://creativecommons.org/licenses/by-nc-sa/3.0/)
%.
% Important note:
% This template needs to be compiled with XeLaTeX.
% The main document font is called Fontin and can be downloaded for free
% from here: http://www.exljbris.com/fontin.html
%
%%%%%%%%%%%%%%%%%%%%%%%%%%%%%%%%%%%%%%%%%

%----------------------------------------------------------------------------------------
%	PACKAGES AND OTHER DOCUMENT CONFIGURATIONS
%----------------------------------------------------------------------------------------

\documentclass[a4paper,10pt]{ctexart} % Default font size and paper size

\usepackage{fontspec} % For loading fonts
\defaultfontfeatures{Mapping=tex-text}
\setmainfont{Helvetica Neue}
\setsansfont{Times New Roman}
\setmonofont{Courier New}
\setCJKmainfont{Noto Sans CJK SC}
\setCJKsansfont{Noto Serif CJK SC}
\setCJKmonofont{Noto Sans Mono CJK SC}

\usepackage{xunicode,xltxtra,url,parskip} % Formatting packages

\usepackage[usenames,dvipsnames]{xcolor} % Required for specifying custom colors

% \usepackage[big]{layaureo} % Margin formatting of the A4 page, an alternative to layaureo can be \usepackage{fullpage}
% To reduce the height of the top margin uncomment: \addtolength{\voffset}{-1.3cm}
\usepackage{fullpage}

\usepackage{hyperref} % Required for adding links	and customizing them
\definecolor{linkcolour}{rgb}{0,0,0} % Link color
\hypersetup{colorlinks,breaklinks,urlcolor=linkcolour,linkcolor=linkcolour} % Set link colors throughout the document

\usepackage{titlesec} % Used to customize the \section command
\titleformat{\section}{\Large\raggedright}{}{0em}{}[\titlerule] % Text formatting of sections
\titlespacing{\section}{0pt}{3pt}{6pt} % Spacing around sections
\titleformat{\subsection}{\large\raggedright}{ -}{0.2em}{} % Text formatting of subsections
\titlespacing{\subsection}{0pt}{0pt}{3pt} % Spacing around subsections

\usepackage{makecell}
\usepackage{array}
\usepackage{tabularx}
\usepackage{longtable}

\usepackage{fontawesome5}

\usepackage{enumitem}
\setenumerate[1]{itemsep=3pt,partopsep=0pt,parsep=0pt,topsep=3pt}
\setitemize[1]{itemsep=3pt,partopsep=0pt,parsep=0pt,topsep=3pt}
\setdescription{itemsep=3pt,partopsep=0pt,parsep=0pt,topsep=3pt}

\begin{document}

\pagestyle{empty} % Removes page numbering

\font\fb=''[cmr10]'' % Change the font of the \LaTeX command under the skills section


%----------------------------------------------------------------------------------------
%	NAME AND CONTACT INFORMATION
%----------------------------------------------------------------------------------------

\begin{center}
    {\LARGE 党浩然} \\
    \faIcon{envelope} \  \href{mailto:danghr@outlook.com}{danghr@outlook.com} \  / \  \faIcon{desktop} \  \href{https://www.danghr.com}{danghr.com} \\
\end{center}


%----------------------------------------------------------------------------------------
%	EDUCATION INFO
%----------------------------------------------------------------------------------------

\section{教育经历}

    \begin{itemize}
        \item \textbf{中国科学院计算技术研究所} \hfill 2022年9月至今\\
        电子信息专业 \, 工程硕士在读 \hfill 中国 \, 北京 \\
        \begin{small}
            导师:范东睿(研究员) \\
            GPA:3.81/4 \quad
            成绩单:\href{https://www.danghr.com/mtzh}{danghr.com/mtzh}
        \end{small}
        \item \textbf{上海科技大学} \hfill 2018年9月至2022年6月\\
        计算机科学与技术专业 \, 工学学士        \hfill 中国 \, 上海 \\
        \begin{small}
            GPA:3.50/4 \quad 专业排名:55/194(28.4\%) \quad
            成绩单:\href{https://www.danghr.com/ugtzh}{danghr.com/ugtzh}
        \end{small}
    \end{itemize}


%----------------------------------------------------------------------------------------
%	RESEARCH WORK
%----------------------------------------------------------------------------------------

\section{科研经历}

    \subsection{科研成果}

        \begin{itemize}
            \item \underline{H Dang}, C Ye, Y Hu, and C Wang, "\textbf{NobLSM: An LSM-tree with Non-blocking Writes for SSDs}," \textit{ACM/IEEE Design Automation Conference (DAC)}, 2022. DOI: \href{https://doi.org/10.1145/3489517.3530470}{10.1145/3489517.3530470}.
            \item D Wang, M Yan, Y Teng, D Han, \underline{H Dang}, X Ye, and D Fan, “\textbf{A Transfer Learning Framework for High-accurate Cross-workload Design Space Exploration of CPU},” \textit{International Conference on Computer Aided Design (ICCAD)}, 2023. DOI: \href{https://doi.org/10.1109/ICCAD57390.2023.10323840}{10.1109/ICCAD57390.2023.10323840}
        \end{itemize}

    \subsection{研究经历}

    \begin{itemize}
        \item \textbf{图计算组} \quad 中国科学院计算技术研究所 \, 高通量计算机研究中心 \hfill 2021年12月至今 \\
        {\small 课题组长:严明玉(副研究员)} \hfill 中国 \, 北京
        \begin{small}
            \begin{itemize}
                \item 完成了一项图神经网络在多GPU架构下加速的研究,并以此获得优秀毕业设计奖项。
            \end{itemize}
        \end{small}
    \end{itemize}

    \begin{itemize}
        \item \textbf{吐司实验室} \quad 上海科技大学 \, 信息科学与技术学院 \hfill 2021年7月至2022年1月 \\
        {\small 实验室负责人:王春东(助理教授、研究员)} \hfill 中国 \, 上海
        \begin{small}
            \begin{itemize}
                \item 负责NobLSM项目中数据库部分的设计与实现,以及性能测试、数据收集等工作。
            \end{itemize}
        \end{small}
    \end{itemize}

    \begin{itemize}
        \item \textbf{态度研究实验室} \quad 上海科技大学 \, 创业与管理学院 \hfill 2020年9月至2021年9月 \\
        {\small 实验室负责人:杨丽凤(副教授、研究员)} \hfill 中国 \, 上海
        \begin{small}
            \begin{itemize}
                \item 参与了《信息科学技术伦理与道德》书籍出版准备,提供出版社比选建议,并参与了样章编写等工作。
            \end{itemize}
        \end{small}
    \end{itemize}


%----------------------------------------------------------------------------------------
%	TEACHING ASSISTANT
%----------------------------------------------------------------------------------------

\section{助教经历}

    \begin{itemize}
        \item \textbf{操作系统I}(本科) \hfill 上海科技大学 \quad 2021-2022学年秋学期
    \end{itemize}


%----------------------------------------------------------------------------------------
%	LANGUAGE AND STANDARDIZED TESTS
%----------------------------------------------------------------------------------------

\section{语言与标化考试成绩}

    \begin{itemize}
        \item \textbf{美国研究生入学考试(GRE)} \quad 328分 \hfill 2022年8月 \\
        \begin{small}
            文字推理:158 \quad 数量推理:170 \quad 分析性写作:4
        \end{small}
        % \item \textbf{托福网考(TOEFL iBT)} \quad 108分 \hfill 2020年12月 \\
        % \begin{small}
        %     阅读:28 \quad 听力:30 \quad 口语:21 \quad 写作:29
        % \end{small}
        \item \textbf{大学英语六级考试(CET-6)} \quad 572分 \hfill 2021年6月 \\
        \begin{small}
            听力:198 \quad 阅读:211 \quad 写作与翻译:163
        \end{small}
    \end{itemize}


\newpage


%----------------------------------------------------------------------------------------
%	CERTIFICATES
%----------------------------------------------------------------------------------------

\section{证书与荣誉称号}

    \begin{itemize}
        \item \textbf{上海科技大学信息学院2022年优秀毕业设计} \hfill \quad 2022年6月
        \item \textbf{上海科技大学2018级本科生产业实践优秀个人称号} \hfill \quad 2021年6月
        \item \textbf{上海科技大学2019年本科生社会实践优秀个人称号} \hfill \quad 2019年11月
    \end{itemize}


%----------------------------------------------------------------------------------------
%	Voluntary Work
%----------------------------------------------------------------------------------------

\section{志愿与社会活动}

    \begin{itemize}
        \item \textbf{“向阳花活动”—海淀医院志愿者} \hfill 中科院计算所 \quad 2023年12月
            \begin{small} \begin{itemize}
                \item 协助医院服务中心解答患者问题、打印病理报告。
            \end{itemize} \end{small}
        \item \textbf{2020年新生入学} \hfill 上海科技大学 \quad 2020年9月
            \begin{small} \begin{itemize}
                \item 负责研究生新生行李提取的引导工作。
            \end{itemize} \end{small}
        \item \textbf{2020年校园开放日} \hfill 上海科技大学 \quad 2020年7月
            \begin{small} \begin{itemize}
                \item 参加了设备检查及试卷下载工作。
            \end{itemize} \end{small}
        \item \textbf{2019年本科招生}  \hfill 上海科技大学 \quad 2019年6月
            \begin{small} \begin{itemize}
                \item 参加了北京招生组的咨询工作。
            \end{itemize} \end{small}
    \end{itemize}

\end{document}

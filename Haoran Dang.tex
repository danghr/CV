%%%%%%%%%%%%%%%%%%%%%%%%%%%%%%%%%%%%%%%%%
% Plasmati Graduate CV
% LaTeX Template
% Version 1.0 (24/3/13)
%
% This template has been downloaded from:
% http://www.LaTeXTemplates.com
%
% Original author:
% Alessandro Plasmati (alessandro.plasmati@gmail.com)
%
% License:
% CC BY-NC-SA 3.0 (http://creativecommons.org/licenses/by-nc-sa/3.0/)
%
% Important note:
% This template needs to be compiled with XeLaTeX.
% The main document font is called Fontin and can be downloaded for free
% from here: http://www.exljbris.com/fontin.html
%
%%%%%%%%%%%%%%%%%%%%%%%%%%%%%%%%%%%%%%%%%

%----------------------------------------------------------------------------------------
%	PACKAGES AND OTHER DOCUMENT CONFIGURATIONS
%----------------------------------------------------------------------------------------

\documentclass[a4paper,10pt]{ctexart} % Default font size and paper size

\usepackage{fontspec} % For loading fonts
\defaultfontfeatures{Mapping=tex-text}
\setmainfont{Helvetica Neue}
\setsansfont{Times New Roman}
\setmonofont{Courier}
\setCJKmainfont{Noto Sans CJK SC}[
    BoldFont=Noto Sans CJK SC Bold,
    ItalicFont=Noto Sans CJK SC,
    BoldItalicFont=Noto Sans CJK SC Bold,
]
\setCJKsansfont{Noto Serif CJK SC}[
    BoldFont=Noto Serif CJK SC Bold,
    ItalicFont=Noto Serif CJK SC,
    BoldItalicFont=Noto Serif CJK SC Bold,
]
\setCJKmonofont{Noto Sans Mono CJK SC}[
    BoldFont=Noto Sans Mono CJK SC Bold,
    ItalicFont=Noto Sans Mono CJK SC,
    BoldItalicFont=Noto Sans Mono CJK SC Bold,
]

\usepackage{xunicode,xltxtra,url,parskip} % Formatting packages

\usepackage[usenames,dvipsnames]{xcolor} % Required for specifying custom colors

% \usepackage[big]{layaureo} % Margin formatting of the A4 page, an alternative to layaureo can be \usepackage{fullpage}
% To reduce the height of the top margin uncomment: \addtolength{\voffset}{-1.3cm}
% \usepackage{fullpage}
\usepackage[hmargin=0.5in, vmargin=0.5in]{geometry}

\usepackage{hyperref} % Required for adding links	and customizing them
\definecolor{linkcolour}{rgb}{0,0,0} % Link color
\hypersetup{colorlinks,breaklinks,urlcolor=linkcolour,linkcolor=linkcolour} % Set link colors throughout the document

\usepackage{titlesec} % Used to customize the \section command
\titleformat{\section}{\Large\raggedright}{}{0em}{}[\titlerule] % Text formatting of sections
\titlespacing{\section}{0pt}{2pt}{4pt} % Spacing around sections
\titleformat{\subsection}{\large\raggedright}{ –~}{0pt}{} % Text formatting of subsections
\titlespacing{\subsection}{0pt}{0pt}{2pt} % Spacing around subsections

\usepackage{makecell}
\usepackage{array}
\usepackage{tabularx}
\usepackage{longtable}

\usepackage{fontawesome5}

\usepackage{enumitem}
\setenumerate[1]{itemsep=2pt,partopsep=0pt,parsep=0pt,topsep=2pt}
\setitemize[1]{itemsep=2pt,partopsep=0pt,parsep=0pt,topsep=2pt}
\setdescription{itemsep=2pt,partopsep=0pt,parsep=0pt,topsep=2pt}

\begin{document}

\pagestyle{empty} % Removes page numbering

\font\fb=''[cmr10]'' % Change the font of the \LaTeX command under the skills section

% Sections that should not be cut in two pages
\newenvironment{keepsection}{\par\noindent\minipage{\textwidth}}{\endminipage\par}

% Tilde in date range
\newcommand{\datetlide}{\textasciitilde \ }


%----------------------------------------------------------------------------------------
%	NAME AND CONTACT INFORMATION
%----------------------------------------------------------------------------------------

\begin{keepsection}
\begin{center}
    \textbf{{\LARGE Haoran Dang}{\large(党浩然)}} \\
    \faIcon{envelope} \  \href{mailto:danghr@outlook.com}{danghr@outlook.com} \  / \  \faIcon{desktop} \  \href{https://www.danghr.com}{danghr.com} \\
\end{center}
\end{keepsection}


%----------------------------------------------------------------------------------------
%	EDUCATION
%----------------------------------------------------------------------------------------
\begin{keepsection}

\section{EDUCATION}

\begin{itemize}
    \item \textbf{University of Chinese Academy of Sciences} \hfill 2022-09 \datetlide Present \\
    \textbf{Institute of Computing Technology, Chinese Academy of Sciences} \hfill Beijing, China \\
    M.Eng. Student in Computer Technology \quad {\small Supervisor: Prof. Dongrui Fan \quad GPA: 3.81/4 \quad Rank: 2/41 (4.9\%)} 
    \item \textbf{ShanghaiTech University} \hfill 2018-09 \datetlide 2022-06 \\
    B.Eng. in Computer Science and Technology \quad {\small GPA: 3.50/4 \quad Rank: 55/194 (28.4\%)} \hfill Shanghai, China \\
\end{itemize}

\end{keepsection}


%----------------------------------------------------------------------------------------
%	RESEARCH & PROJECTS
%----------------------------------------------------------------------------------------

\begin{keepsection}
\section{RESEARCH \& PROJECTS}

\subsection{Publications}
    \begin{enumerate}
        \item \label{GDL-GNN-Paper} \underline{H Dang}, M Wu, M Yan, X Ye, and D Fan, ``\textbf{GDL-GNN: Applying GPU Dataloading of Large Datasets for Graph Neural Network Inference},'' \textit{Euro-Par 2024}.
        \item \label{NobLSM-Paper} \underline{H Dang}, C Ye, Y Hu, and C Wang, ``\textbf{NobLSM: An LSM-tree with Non-blocking Writes for SSDs},'' \textit{DAC 2022}.
        \item \label{TrEnDSE-Paper} D Wang, M Yan, Y Teng, D Han, \underline{H Dang}, X Ye, and D Fan, ``\textbf{A Transfer Learning Framework for High-accurate Cross-workload Design Space Exploration of CPU},'' \textit{ICCAD 2023}.
    \end{enumerate}
\end{keepsection}

\begin{keepsection}
\subsection{Research}
\begin{itemize}
    \item \textbf{Lab of Graph Intelligent Machine} (GIMLab) \quad HTCRC, ICT, CAS \hfill 2021-12 \datetlide Present \\
    {\small Director: Assoc. Prof. Mingyu Yan} \hfill Beijing, China
    \begin{small}
        \begin{itemize}
            \item Currently conducting research to accelerate the mini-batch inference process for GNNs on GPU platforms.
            \item Conducted research to accelerate full-graph GNN accurate inference on multi-GPU platforms. Developed and implemented GDL-GNN, an optimized GPU data-loading method based on graph partitioning to minimize data accesses outside GPU memory, resulting in a 59.9\% performance improvement over the baseline. (See Publication \ref{GDL-GNN-Paper})
            \item Developed an undergraduate FYP on optimizing GNN training, halving training times by implementing a multithreaded dataloader. Awarded the SIST Outstanding Thesis.
        \end{itemize}
    \end{small}

    \item \textbf{Toast Lab} \quad SIST, ShanghaiTech \hfill 2021-07 \datetlide 2022-02 \\
    {\small PI: Assist. Prof. Chundong Wang} \hfill Shanghai, China
    \begin{small}
        \begin{itemize}
            \item Designed and implemented NobLSM's database component, utilizing a modified ext4 to minimize costly \texttt{sync} operations by delaying deletions of old SSTables until the corresponding compacted SSTables have been automatically flushed onto the disk. Demonstrated enhanced performance via benchmarking against SOTA LSM-trees. (See Publication \ref{NobLSM-Paper})
        \end{itemize}
    \end{small}

    \item \textbf{Attitude Research Lab} (ARLab) \quad SEM, ShanghaiTech \hfill 2020-09 \datetlide 2021-09 \\
    {\small PI: Assoc. Prof. Lifeng Yang} \hfill Shanghai, China
    \begin{small}
        \begin{itemize}
            \item Participated in preparing the book \textit{Ethics and Morality in Information Science and Technology}. Provided suggestions on selecting publishing houses and participated in writing example chapters.
        \end{itemize}
    \end{small}
\end{itemize}
\end{keepsection}

\begin{keepsection}
\subsection{Projects}
\begin{itemize}
    \item \textbf{The ``One Student One Chip'' (一生一芯) Project} \hfill 2024-01 \datetlide Present
    \begin{small}
        \begin{itemize}
            \item Student ID: ysyx\_24070014 \quad Learning record: \href{https://www.danghr.com/ysyx/record}{danghr.com/ysyx/record} \quad Code: \href{https://www.danghr.com/ysyx/code}{danghr.com/ysyx/code}
            \item Software: Implemented NEMU, a RISC-V emulator for teaching purposes, supporting RV32IM instructions up to now.
            \item Hardware: Currently developing a single-cycle RISC-V CPU using Verilog.
        \end{itemize}
    \end{small}
    \item \textbf{Pintos} {\small (As the course project of Operating System I in ShanghaiTech)} \hfill 2020-09 \datetlide 2021-01
    \begin{small}
        \begin{itemize}
            \item Implemented a multi-threading kernel with scheduling \& thread management, user program \& system calls, virtual memory, and file system with buffer cache.
        \end{itemize}
    \end{small}
\end{itemize}
\end{keepsection}

\begin{keepsection}
\subsection{TEACHING}

    \begin{itemize}
        \item \textbf{Operating Systems I} (TA) \hfill ShanghaiTech \quad Fall, AY 2021-2022
    \end{itemize}
\end{keepsection}


%----------------------------------------------------------------------------------------
%	SKILLS
%----------------------------------------------------------------------------------------

\begin{keepsection}

\section{SKILLS}
\begin{itemize}
    \item \textbf{Programming Languages \& Frameworks:} Proficient in C/C++ and Python, with knowledge in PyTorch, DGL, Verilog, Shell scripting, and \LaTeX.
    \item \textbf{Languages:} Mandarin Chinese (native), English (fluent)
        \begin{itemize}
            \item \textbf{TOEFL iBT} \quad Score: 107 {\small (Reading: 29 \, Listening: 27 \, Speaking: 23 \, Writing: 28)} \hfill 2024-04
            \item \textbf{GRE General Test} \quad Score: 328 {\small (Verbal: 158 \, Quantitative: 170 \, Analytical Writing: 4)} \hfill 2022-08
            % \item \textbf{College English Test Band 6} \quad Score: 572 {\small (Listening: 198 \, Reading: 211 \, Writing: 163)} \hfill 2021-06
        \end{itemize}
\end{itemize}

\end{keepsection}


%----------------------------------------------------------------------------------------
%	CERTIFICATES
%----------------------------------------------------------------------------------------

\begin{keepsection}

\section{CERTIFICATES \& AWARDS}
\begin{itemize}
    \item \textbf{Outstanding Student} (AY 2023-2024) \hfill UCAS \quad 2024-05
    \item \textbf{2022 SIST Outstanding Thesis} (for undergraduate final year project) \hfill ShanghaiTech \quad 2022-06
    \item \textbf{Excellent Individual of Industrial Practice in Class of 2018} \hfill ShanghaiTech \quad 2021-06
    \item \textbf{Excellent Individual of Social Practice in 2019} \hfill ShanghaiTech \quad 2019-11
\end{itemize}

\end{keepsection}


%----------------------------------------------------------------------------------------
%	Service
%----------------------------------------------------------------------------------------

\begin{keepsection}

\section{VOLUNTARY SERVICE}
\begin{itemize}
    \item \textbf{20th Public Science Day of ICT, CAS} \quad {\small On-site camera operator for livestream.} \hfill ICT, CAS \quad 2024-05
    \item \textbf{``Sunflower Activity'' - Haidian Hospital Volunteers} \quad {\small Assistance in the Service Center} \hfill ICT, CAS \quad 2023-12
    \item \textbf{2020 Freshmen Enrollment} \quad {\small Assisted in baggage claim for new graduate students.} \hfill ShanghaiTech \quad 2020-09
    \item \textbf{2020 Campus Opening Day} \quad {\small Participated in device check and exam download.} \hfill ShanghaiTech \quad 2020-07
    \item \textbf{2019 Admission Consultation} \quad {\small Participated in admission consultation in Beijing.} \hfill ShanghaiTech \quad 2019-06
\end{itemize}

\end{keepsection}

\end{document}
